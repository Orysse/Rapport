% Glossaire concis pour le rapport de stage TRUST
% Acronymes techniques
\newacronym{ci}{CI}{Continuous Integration}
\newacronym{cd}{CD}{Continuous Deployment}
\newacronym{gpu}{GPU}{Graphics Processing Unit}
\newacronym{cpu}{CPU}{Central Processing Unit}
\newacronym{hpc}{HPC}{High Performance Computing}
\newacronym{api}{API}{Application Programming Interface}
\newacronym{s3}{S3}{Simple Storage Service}

% Organisations et services
\newacronym{epita}{EPITA}{École Pour l'Informatique et les Techniques Avancées}
\newacronym{cea}{CEA}{Commissariat à l'Énergie Atomique et aux énergies alternatives}
\newacronym{des}{DES}{Direction des Énergies}
\newacronym{dam}{DAM}{Direction des Applications Militaires}
\newacronym{drt}{DRT}{Direction de la Recherche Technologique}
\newacronym{drf}{DRF}{Direction de la Recherche Fondamentale}
\newacronym{isas}{ISAS}{Institut des Sciences Appliquées et de la Simulation}
\newacronym{dm2s}{DM2S}{Département de Modélisation des Systèmes et des Structures}
\newacronym{lcan}{LCAN}{Laboratoire de Calcul Avancé Neutronique}
\newacronym{sgls}{SGLS}{Service de Génie Logiciel pour la Simulation}
\newacronym{rsi}{RSI}{Réseaux et Systèmes d'Information}
\newacronym{drt}{DRT}{Direction de la Recherche Technologique}

% Technologies et outils
\newacronym{yaml}{YAML}{YAML Ain't Markup Language}
\newacronym{lfs}{LFS}{Large File Storage}
\newacronym{uid}{UID}{User Identifier}
\newacronym{cuda}{CUDA}{Compute Unified Device Architecture}
\newacronym{rocm}{ROCm}{Radeon Open Compute}
\newacronym{ubi}{UBI}{Universal Base Image}
\newacronym{xml}{XML}{eXtensible Markup Language}

% Termes spécifiques
\newglossaryentry{trust}{
  name={TRUST},
  description={Thermohydraulics Research in Unstructured Solvers and Technologies - logiciel open-source de thermohydraulique développé par le CEA}
}
\newglossaryentry{baltiks}{
  name={Baltiks},
  description={Ensemble de projets satellites qui dépendent de TRUST et étendent ses fonctionnalités pour des applications spécifiques}
}
\newglossaryentry{gitops}{
  name={GitOps},
  description={Méthodologie de développement logiciel qui utilise Git comme source unique de vérité pour l'infrastructure et les applications}
}
\newglossaryentry{pipeline}{
  name={Pipeline},
  description={Ensemble automatisé de processus permettant de compiler, tester et déployer du code}
}
\newglossaryentry{runner}{
  name={Runner},
  description={Agent d'exécution GitLab qui prend en charge l'exécution des jobs définis dans les pipelines CI/CD}
}
\newglossaryentry{rootless}{
  name={Rootless},
  description={Mode d'exécution de conteneurs sans privilèges administrateur (root), améliorant la sécurité}
}
\newglossaryentry{artifact}{
  name={Artifact},
  description={Fichier ou ensemble de fichiers produits par un job CI/CD et conservés pour usage ultérieur}
}
\newglossaryentry{docker}{
  name={Docker},
  description={Plateforme de conteneurisation permettant d'empaqueter et d'exécuter des applications dans des environnements isolés}
}
\newglossaryentry{exascale}{
  name={Exascale},
  description={Classe de supercalculateurs capables d'effectuer au moins un exaflop (10$^{18}$) d'opérations par seconde}
}
\newglossaryentry{devops}{
    name={DevOps},
    description={Pratiques unifiant le développement logiciel (Dev) et l'administration des infrastructures (Ops)}
}

% Nouvelles entrées ajoutées
\newglossaryentry{gitlab}{
  name={GitLab},
  description={Plateforme DevOps intégrée offrant la gestion de dépôts Git, l'intégration continue et le déploiement continu}
}
\newglossaryentry{dockercompose}{
  name={Docker Compose},
  description={Outil pour définir et orchestrer des applications multi-conteneurs via des fichiers YAML}
}
\newglossaryentry{slirp4netns}{
  name={slirp4netns},
  description={Outil permettant la mise en réseau de conteneurs en espace utilisateur sans privilèges réseau}
}
\newglossaryentry{minio}{
  name={MinIO},
  description={Serveur de stockage objet open-source compatible avec l'API Amazon S3, utilisé pour le cache distribué}
}
\newglossaryentry{prometheus}{
  name={Prometheus},
  description={Système open-source de surveillance et d'alerte collectant et stockant des métriques en séries temporelles}
}
\newglossaryentry{grafana}{
  name={Grafana},
  description={Plateforme open-source de visualisation et d'analyse permettant de créer des tableaux de bord à partir de métriques}
}
\newglossaryentry{node-exporter}{
  name={Node Exporter},
  description={Outil de monitoring de l'écosystème Prometheus qui exporte des métriques relatives au matériel et au système d'exploitation de l'hôte (CPU, mémoire, disque, réseau)}
}
\newglossaryentry{cadvisor}{
  name={cAdvisor},
  description={Container Advisor - outil de Google permettant de collecter, d'agréger et d'exporter des métriques sur l'utilisation des ressources et les performances des conteneurs en cours d'exécution}
}
\newglossaryentry{mergerequest}{
  name={Merge Request},
  description={Proposition de modification du code source permettant la revue et la validation avant intégration dans la branche principale}
}
\newglossaryentry{cache}{
  name={Cache},
  description={Mécanisme de stockage temporaire de données fréquemment utilisées pour accélérer les exécutions ultérieures}
}
\newglossaryentry{subuid}{
  name={subuid/subgid},
  description={Plages d'identifiants utilisateurs/groupes subordonnés permettant le remapping d'UID dans les conteneurs rootless}
}
\newglossaryentry{externalpackages}{
  name={ExternalPackages},
  description={Dépôt contenant l'ensemble des dépendances et bibliothèques tierces nécessaires à la compilation de TRUST}
}
\newglossaryentry{atelier}{
  name={L'Atelier},
  description={Système de tests automatisés historique de TRUST basé sur des scripts Bash s'exécutant nocturnement}
}
\newglossaryentry{junit}{
  name={JUnit},
  description={Format XML standardisé pour reporter les résultats de tests, permettant leur intégration dans les outils CI/CD}
}
