\section{Introduction}

Le présent rapport expose les travaux réalisés au cours du stage de fin
de tronc commun de l'\textit{Ecole Pour l'Informatique et les Techniques Avancées}
(EPITA) à Paris, du 1 septembre 2025 au 30 janvier 2026. Il a été réalisé au
\textit{Commisariat à l'Energie Atomique et aux énergies alternatives} (CEA)
Paris-Saclay, plus particulièrement dans le \textit{Service de Génie Logiciel
pour la Simulation} (SGLS), au sein du \textit{Laboratoire de Calcul Avancé
Neutronique} (LCAN), sous la tutelle de M.~Bourgeois et M.~Bruneton.

\bigskip

L'objectif principal du stage a été de moderniser et d'industrialiser
l'infrastructure d'intégration continue (CI) de TRUST, un logiciel open-source
de thermohydraulique développé par le CEA, ainsi que de l'ensemble de
l'écosystème de projets qui en dépendent. Mon travail visait à mettre en place
des pipelines de CI robustes et automatisés, capables de gérer les
spécificités des architectures GPU et des environnements de calcul haute
performance. Le but était d'assurer la fiabilité et la reproductibilité des
tests, tout en respectant les contraintes de sécurité élevées imposées par le
contexte du CEA. Cette approche permettrait non seulement d'améliorer la
qualité du code et de détecter les régressions plus rapidement, mais aussi de
faciliter le déploiement continu des nouvelles fonctionnalités, rendant ainsi
possible l'évolution rapide et sécurisée de l'écosystème TRUST. S'inscrivant
dans le développement des machines exascale en Europe et en France, la mise en
place de tests automatisés et l'orchestration de campagnes de validation
constituèrent une partie importante de ma mission. Ces tests permettaient de
garantir la stabilité des logiciels sur des architectures hybrides CPU/GPU et
de s'assurer que les performances et la précision des simulations restaient
optimales.

\bigskip

Mon choix pour ce stage au CEA s'appuie aussi sur la réputation de
l'établissement en matière de recherche appliquée et sur la possibilité de
contribuer à un projet open-source, s'appuyant sur des technologies DevOps de
pointe dans un contexte exigeant. Le CEA est reconnu pour son excellence dans
le domaine de la recherche scientifique et technologique, et travailler dans
un tel environnement représentait une opportunité unique d'acquérir des
compétences avancées en DevOps et en gestion d'infrastructures de calcul haute
performance, tout en collaborant avec des experts. Enfin, la perspective de
travailler dans un environnement de haute sécurité et de gérer des
infrastructures complexes a fortement encouragé ma décision. Contribuer à un
projet open-source d'envergure utilisé par la communauté scientifique
représentait une opportunité unique pour approfondir mes connaissances et pour
participer à des projets à la pointe de la technologie.

\bigskip

Les travaux effectués au cours de ce stage se sont structurés en cinq phases.
Chacune de ces parties a fait l'objet d'un rapport qui détaille mon
cheminement dans la mise en place et l'optimisation de l'infrastructure DevOps.

\medskip

\begin{enumerate}

\item Découverte de l'écosystème TRUST et de son architecture logicielle.
    Familiarisation avec les contraintes de sécurité et les spécificités des
    environnements de calcul haute performance du CEA.

\medskip

\item Modernisation de la CI de TRUST avec mise en place de pipelines automatisés
    pour les tests sur architectures CPU et GPU. Intégration des mécanismes de
    validation et de tests de non-régression.

\medskip

\item Extension de l'infrastructure de CI à l'ensemble des projets dépendants de
    TRUST. Optimisation des temps de build et de test. Mise en place de mécanismes
    de monitoring et de reporting.

\medskip

\item Rédaction d'un rapport et d'une documentation technique sur l'infrastructure
    mise en place et les procédures de maintenance.

\end{enumerate}
