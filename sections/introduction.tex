\section{Introduction}

Le présent rapport expose les travaux réalisés au cours du stage de fin
de tronc commun de l'\gls{epita} à Paris, du 1 septembre 2025 au 30 janvier 2026.
Il a été réalisé au \gls{cea} Paris-Saclay, plus particulièrement dans le
\gls{sgls}, au sein du \gls{lcan}, sous la tutelle de M.~Bourgeois et M.~Bruneton,
tout deux ingénieur-chercheur au \gls{lcan}.

\bigskip

L'objectif principal du stage a été de moderniser et d'industrialiser
l'infrastructure de \gls{ci} de \gls{trust}, un logiciel open-source
de thermohydraulique développé par le \gls{cea}, ainsi que de l'ensemble de
l'écosystème de projets qui en dépendent. Mon travail visait à mettre en place
des pipelines de \gls{ci} robustes et automatisés, capables de gérer les
spécificités des architectures \gls{gpu} et des environnements de \gls{hpc}.
Le but était d'assurer la fiabilité et la reproductibilité des
tests, tout en respectant les contraintes de sécurité élevées imposées par le
contexte du \gls{cea}. Cette approche permettrait non seulement d'améliorer la
qualité du code et de détecter les régressions plus rapidement, mais aussi de
faciliter le déploiement continu des nouvelles fonctionnalités, rendant ainsi
possible l'évolution rapide et sécurisée de l'écosystème \gls{trust}. S'inscrivant
dans le développement des machines \gls{exascale} en Europe et en France, la mise en
place de tests automatisés et l'orchestration de campagnes de validation
constituèrent une partie importante de ma mission. Ces tests permettaient de
garantir la stabilité des logiciels sur des architectures hybrides \gls{cpu}/\gls{gpu} et
de s'assurer que les performances et la précision des simulations restaient
optimales.

\bigskip

Mon choix pour ce stage au \gls{cea} s'appuie aussi sur la réputation de
l'établissement en matière de recherche appliquée et sur la possibilité de
contribuer à un projet open-source, s'appuyant sur des technologies \gls{devops} de
pointe dans un contexte exigeant. Le \gls{cea} est reconnu pour son excellence dans
le domaine de la recherche scientifique et technologique, et travailler dans
un tel environnement représentait une opportunité unique d'acquérir des
compétences avancées en \gls{devops} et en gestion d'infrastructures de \gls{hpc},
tout en collaborant avec des experts. Enfin, la perspective de
travailler dans un environnement de haute sécurité et de gérer des
infrastructures complexes a fortement encouragé ma décision. Contribuer à un
projet open-source d'envergure utilisé par la communauté scientifique
représentait une opportunité unique pour approfondir mes connaissances et pour
participer à des projets à la pointe de la technologie.

\bigskip

Les travaux effectués au cours de ce stage se sont structurés en quatre phases.
Chacune de ces parties a fait l'objet d'un rapport qui détaille mon
cheminement dans la mise en place et l'optimisation de l'infrastructure \gls{devops}.

\medskip

\begin{enumerate}

\item Découverte de l'écosystème \gls{trust} et de son architecture logicielle.
    Familiarisation avec les contraintes de sécurité et les spécificités des
    environnements de \gls{hpc} du \gls{cea}.

\medskip

\item Modernisation de la \gls{ci} de \gls{trust} avec mise en place de pipelines automatisés
    pour les tests sur architectures \gls{cpu} et \gls{gpu}. Intégration des mécanismes de
    validation et de tests de non-régression.

\medskip

\item Extension de l'infrastructure de \gls{ci} à l'ensemble des projets dépendants de
    \gls{trust}. Optimisation des temps de build et de test. Mise en place de mécanismes
    de monitoring et de reporting.

\medskip

\item Rédaction d'un rapport et d'une documentation technique sur l'infrastructure
    mise en place et les procédures de maintenance.

\end{enumerate}
