\section{Présentation de l'entreprise}

\subsection{Le secteur d'activité}

Le \gls{cea} est un acteur clef de la recherche en France. La recherche scientifique
est un secteur dans lequel les travaux menés sont caractérisés par leur
ambitions, leur complexité et leur innovation technologique. Ils sont
entrepris pour répondre à de grands défis industriels (exploitation de l'
énergie nucléaire dans le civil et le militaire) et sociétaux (transition
écologique, stockage de l'énergie) et impliquent de nombreuses collaborations
avec des partenaires académiques ou industriels.
Les missions du \gls{cea} s'articulent autour de différents axes :

\medskip
\begin{itemize}
\item[•] Défense et sécurité nationale : Il est chargé de répondre aux enjeux
    de la dissuasion nucléaire (renouvellement des armes nucléaires et des
    chaufferies nucléaires de propulsion navale, lutte contre la prolifération
    nucléaire).

\medskip

\item[•] Énergies : Le \gls{cea} est au premier plan de la transition énergétique.
    Des recherches sur les façons de produire de l'énergie bas carbone y sont
    menées.

\medskip

\item[•] Transition numérique : L'expertise du \gls{cea} dans les domaines de l'
    électronique et du numérique lui permette de concevoir des plateformes
    technologiques innovantes en micro et nano-électronique, en robotique, en
    intelligence artificielle et en technologies quantiques.

\medskip

\item[•] Recherche fondamentale : L'institution investit dans le recherche
    théorique d'excellence, aussi bien dans les domaines de la physique, de l'
    astrophysique, des sciences des matériaux, de la chimie, de la biologie et
    de la santé.

\medskip

\item[•] Assainissement et démantèlement des installations nucléaires en fin
    de vie : Le \gls{cea} est un expert de la gestion des déchets nucléaires.

\medskip

\item[•] Technologies de la santé : L'organisme est particulièrement actif dans le
    domaine de l'imagerie médicale, avec entre autre le développement d'IRM à
    très haut champ.
\end{itemize}

\subsection{L'entreprise}

Le \gls{cea} est un établissement à caractère scientifique, technique et industriel. Il
est placé sous la tutelle des ministres chargés de l'énergie, de la recherche,
de l'industrie et de la défense. Fin 2022, il emploie plus de 21 000 salariés,
pour un budget annuel de 5,8 milliards d'euros. Le \gls{cea} a pour mission
principale de développer les applications de l'énergie nucléaire dans les
domaines scientifique, industriel et de la sécurité nationale.

\medskip

Avec ces 9 centres de recherche, le \gls{cea} valorise les technologies qu'il
développe et les transfère vers l'industrie en soutenant ainsi la
compétitivité et la souveraineté des entreprises technologiques françaises. Il
est ainsi un véritable moteur de l'innovation. Le \gls{cea} en quelques chiffres :

\medskip
\begin{itemize}
\item[•] 1\textsuperscript{er} déposant français de brevets en Europe
\item[•] 6 980 familles de brevets actives
\item[•] 700 partenaires industriels
\item[•] > 5000 publications
\end{itemize}

\medskip
\noindent Afin de répondre à ses missions, le \gls{cea} se décompose en quatre
directions opérationnelles :

\medskip
\begin{itemize}
  \item[•] \gls{des}
  \item[•] \gls{dam}
  \item[•] \gls{drt}
  \item[•] \gls{drf}
\end{itemize}

\clearpage
\subsection{Le service}

Le \gls{sgls} est un service de la \acrshort{des}, plus précisément de l'\textit{\gls{isas} pour les énergies bas carbone},
dans le \textit{\gls{dm2s}}. Cette entité du \gls{cea} se compose de 40 collaborateurs et 20 étudiants. Il
conçoit, développe, distribue et maintient des outils informatiques et
plateformes logicielles dont les logiciels open-source SALOME, \gls{trust} et Uranie,
qui représentent plus de 100 000 téléchargements/an. Ses domaines de
recherche sont :

\medskip
\begin{itemize}
  \item[•] Conception Assistée par Ordinateur
  \item[•] \gls{hpc}
  \item[•] Intelligence Artificielle
  \item[•] Science des données
  \item[•] Maillages
  \item[•] Interfaces Homme Machine
  \item[•] Méthodes numériques
  \item[•] Modélisation physique
  \item[•] Quantification des incertitudes
\end{itemize}
\medskip

Le \gls{sgls} fournit des outils open-source utilisés par les équipes métiers du \gls{cea}
pouvant s'appliquer en neutronique, thermohydraulique, mécanique, science des
matériaux que ce soit dans le cadre du nucléaire, du spatial ou des nouvelles
technologies de l'énergie. Sur l'ensemble de ces activités, le \gls{sgls} ambitionne
de proposer une interface entre les communautés scientifiques académiques et
le monde de l'industrie. Grâce à la présence permanente de doctorants et de
chercheurs habilités à diriger des recherches, le \gls{sgls} est alimenté par une
veille technique active pour un suivi de l'état de l'art.

\subsection{Le positionnement du stage dans les travaux de l'entreprise}

Le déploiement des machines \gls{exascale} en Europe ouvre une nouvelle ère dans le
\gls{hpc}, avec des capacités de calcul inédites. Acteur de
premier plan dans ce domaine, le \gls{cea} accompagne cette évolution en intégrant
les \glspl{gpu} au sein des architectures de calcul, en complément des \glspl{cpu} multi-cœurs
traditionnellement utilisés. Cette architecture hybride, alliant \gls{cpu} pour leur
polyvalence et \gls{gpu} pour leur puissance de calcul parallèle, impose de nouveaux
défis, notamment en matière de gestion énergétique, de consommation mémoire et
d'optimisation des échanges entre ces processeurs.

\bigskip

Les codes de calcul scientifique, qui occupent une place centrale dans des
applications comme la thermohydraulique ou la mécanique des fluides, doivent s'
adapter à cette transformation vers le calcul hybride. Néanmoins, cette
évolution impose des contraintes strictes en matière de validation, de tests
et de déploiement continu, d'autant plus dans un environnement où la sécurité
et la fiabilité des résultats sont primordiales. C'est précisément dans ce
contexte que s'inscrit mon stage, où j'ai pour mission d'assurer l'intégration
et le déploiement continu de ces outils de simulation sur des infrastructures
de calcul avancées.

\bigskip

Au sein du \gls{lcan}, mon travail se concentre sur la gestion de la \gls{ci}
de \gls{trust}, un logiciel open-source de thermohydraulique développé
par le \gls{cea}. Ce dernier est un environnement dédié aux calculs intensifs et à
la simulation scientifique. Mon rôle consiste à mettre en place et maintenir
des \glspl{pipeline} de \gls{ci} adaptés aux architectures \gls{gpu}, à orchestrer les tests sur
supercalculateurs dans un contexte de haute sécurité, et à assurer la \gls{ci} de l'
ensemble des projets dépendant de \gls{trust}. Ce travail contribue directement aux
recherches du \gls{cea} pour garantir la fiabilité et les performances des outils de
simulation tout en répondant aux exigences de sécurité des infrastructures de
calcul de prochaine génération.
