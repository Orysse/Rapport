\section{Présentation de l'entreprise}

\subsection{Le secteur d'activité}

Le CEA est un acteur clef de la recherche en France. La recherche scientifique est un secteur dans lequel les travaux menés sont caractérisés par leur ambitions, leur complexité et leur innovation technologique. Ils sont entrepris pour répondre à de grands défis industriels (exploitation de l'énergie nucléaire dans le civil et le militaire) et sociétaux (transition écologique, stockage de l'énergie) et impliquent de nombreuses collaborations avec des partenaires académiques ou industiels.
Les missions du CEA s'articulent autour de différents axes:

\medskip
\begin{itemize}
\item[•] Défense et sécurité nationale: Il est chargé de répondre aux enjeux de la dissuasion nucléaire (renouvellement des armes nucléaires et des chaufferies nucléaires de propulsion navale, lutte contre la prolifération nucléaire).

\medskip

\item[•] Energies: Le CEA est au premier plan de la transition énergétique. Des recherches sur les façons de produire de l'énergie bas carbone y sont menées.

\medskip

\item[•] Transition numérique: L'expertise du CEA dans les domaines de l'électronique et du numérique lui permette de concevoir des plateformes technologiques innovantes en micro et nano-électronique, en robotique, en intelligence artificielle et en technologies quantiques.

\medskip

\item[•] Recherche fondammental: L'institution investit dans le recherche théorique d'excellence, aussi bien dans les domaines de la physique, de l'astrophysique, des sciences des matériaux, de la chimie, de la biologie et de la santé.

\medskip

\item[•] Assainissement et démantèlement des installations nucléaires en fin de vie: Le CEA est un expert de la gestion des déchets nucléaires.

\medskip

\item[•] Technologies de la santé: L'organisme est particulièrement actif dans le domaine de l'imagerie médicale, avec entre autre le développement d'IRM à très haut champ.
\end{itemize}

\subsection{L'entreprise}

Le CEA est un établissement à caractère scientifique, technique et industriel. Il est placé sous la tutelle des ministres chargés de l'énergie, de la recherche, de l'industrie et de la défense. Fin 2022, il emploie plus de 21 000 salariés, pour un budget annuel de 5,8 milliards d'euros. Le CEA a pour mission principale de développer les applications de l'énergie nucléaire dans les domaines scientifique, industiel et de la sécutité nationale.

\medskip

Avec ces 9 centres de recherche, le CEA valorise les technologies qu'il développe et les transfère vers l'industrie en soutenant ainsi la compétitivité et la souveraineté des entreprises technologiques françaises. Il est ainsi un véritable moteur de l'innovation. Le CEA en quelques chiffres:

\medskip
\begin{itemize}
\item[•] 1\textsuperscript{er} déposant français de brevets en Europe
\item[•] 6 980 familles de brevets actives
\item[•] 700 partenaires industriels
\item[•] > 5000 publications
\end{itemize}

\medskip
\noindent Afin de répondre à ses missions, le CEA se décompose en quatre directions opérationnelles:

\medskip
\begin{itemize}
  \item[•] Direction des énergies (DES)
  \item[•] Direction des applications militaire (DAM)
  \item[•] Direction de la recherche technologique (DRT)
  \item[•] Direction de la recherche fondamentale (DRF)
\end{itemize}

\clearpage
\subsection{Le service}

Le SGLS est une service de la DES, plus précisément de l'\textit{Institut des Sciences Appliquées et de la Simulation pour les énergies bas carbone} (ISAS), dans le \textit{Département de Modélisation des Systèmes et des Structures} (DM2S). Cet entité du CEA se compose de 40 collaborateurs et 20 étudiants. Il conçoit, développe, distribue et maintient des outils informatiques et plateformes logicielles dont les logiciels open-source SALOME, TRUST et Uranie, qui représentent plus de 100 000 téléchargement/an. Ses domaines de recherche sont:

\medskip
\begin{itemize}
  \item[•] Conception Assistée par Odinateur
  \item[•] Calcul Haute Perfomance
  \item[•] Intelligence Artificielle
  \item[•] Science des données
  \item[•] Maillages
  \item[•] Interfaces Homme Machine
  \item[•] Méthodes numériques
  \item[•] Modélisation physique
  \item[•] Quantification des incertitudes
\end{itemize}
\medskip

Le SGLS fournit des outils open-source utilisés par les équipes métiers du CEA pouvant s'appliquer en neutronique, thermohydraulique, mécanique, science des matériaux que ce soit dans le cadre du nucléaire, du spatial ou des nouvelles technologies de l'énergie. Sur l'ensemble de ces activités, le SGLS ambitionne de proposer une interface entre les communautés scientifiques académiques et le monde de l'industrie. Grâce à la présence permanente de doctorants et de chercheurs habilités à diriger des recherche, le SGLS est alimenté par une veille technique active pour un suvi de l'état de l'art.

\subsection{Le positionnement du stage dans les travaux de l'entreprise}

Le déploiement des machines exascale en Europe ouvre une nouvelle ère dans le calcul haute performance, avec des capacités de calcul inédites. Acteur de premier plan dans ce domaine, le CEA accompagne cette évolution en intégrant les GPU au sein des architectures de calcul, en complément des CPU multi-cœurs traditionnellement utilisés. Cette architecture hybride, alliant CPU pour leur polyvalence et GPU pour leur puissance de calcul parallèle, impose de nouveaux défis, notamment en matière de gestion énergétique, de consommation mémoire et d'optimisation des échanges entre ces processeurs.

\bigskip

Les codes de calcul scientifique, qui occupent une place centrale dans des applications comme la thermohydraulique ou la mécanique des fluides, doivent s'adapter à cette transformation vers le calcul hybride. Néanmoins, cette évolution impose des contraintes strictes en matière de validation, de tests et de déploiement continu, d'autant plus dans un environnement où la sécurité et la fiabilité des résultats sont primordiales. C'est précisément dans ce contexte que s'inscrit mon stage, où j'ai pour mission d'assurer l'intégration et le déploiement continu de ces outils de simulation sur des infrastructures de calcul avancées.

\bigskip

Au sein du LCAN, mon travail se concentre sur la gestion de l'intégration continue (CI) de TRUST, un logiciel open-source de thermohydraulique développé par le CEA. Ce dernier est un environnement dédié aux calculs intensifs et à la simulation scientifique. Mon rôle consiste à mettre en place et maintenir des pipelines de CI adaptés aux architectures GPU, à orchestrer les tests sur supercalculateurs dans un contexte de haute sécurité, et à assurer la CI de l'ensemble des projets dépendant de TRUST. Ce travail contribue directement aux recherches du CEA pour garantir la fiabilité et les performances des outils de simulation tout en répondant aux exigences de sécurité des infrastructures de calcul de prochaine génération.
