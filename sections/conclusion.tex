\section{Conclusion générale}

Ce stage avait pour objectif de moderniser l'infrastructure de tests de TRUST en mettant en place une solution d'intégration continue automatisée, capable de valider le code sur multiples architectures et de tester l'ensemble de l'écosystème des projets satellites (Baltiks). Les différents travaux effectués au cours du stage ont permis d'atteindre cet objectif.

\bigskip

Une infrastructure complète d'intégration continue a été développée et déployée, permettant d'automatiser l'ensemble du cycle de validation de TRUST, de la compilation aux tests en passant par la vérification de non-régression des Baltiks. Cette infrastructure a été pensée pour être modulaire, extensible et adaptable aux contraintes de sécurité strictes du CEA. Grâce à cette conception, elle prend en charge cinq distributions Linux pour les tests CPU et trois configurations GPU basées sur NVIDIA CUDA, tout en offrant une architecture facilement extensible pour l'ajout de nouvelles plateformes. L'ensemble du travail réalisé constitue donc une avancée importante vers l'adoption des pratiques GitOps modernes dans le développement de TRUST.

\bigskip

Les résultats obtenus sont très encourageants et démontrent l'intérêt indéniable de l'automatisation des tests dans un projet de cette envergure. Les pipelines mises en place ont permis de réduire drastiquement le temps de feedback pour les développeurs, passant de 12--24 heures avec l'Atelier à 30--60 minutes avec GitLab CI pour une validation complète CPU. Le système de cache à deux niveaux développé pendant le stage permet de diviser par cinq les temps de compilation (de 45--60 minutes à 8--12 minutes avec cache). Toutefois, l'infrastructure conçue nécessite encore des optimisations et un approfondissement au niveau de son déploiement, afin d'exploiter pleinement les capacités des architectures de calcul disponibles. L'un des axes d'amélioration futurs consistera à intégrer le support des architectures GPU AMD via ROCm et à déployer des runners sur les supercalculateurs Topaze et Adastra pour les tests à très grande échelle. De plus, l'expérience acquise au cours de ce stage fournit une base solide pour le déploiement d'infrastructures CI/CD similaires sur d'autres projets du laboratoire, en s'appuyant sur les solutions techniques validées (Docker rootless, Docker Compose, stratégies de cache). Par ailleurs, la nature open-source potentielle de cette infrastructure constitue un atout majeur, offrant un cadre propice au partage de bonnes pratiques DevOps au sein de la communauté du calcul scientifique. Ce projet permet également d'illustrer un cas concret d'intégration de GitLab CI/CD dans un environnement de développement scientifique hautement contraint, qui pourrait servir de référence pour d'autres équipes du CEA.

\bigskip

Ce stage a représenté une opportunité unique pour renforcer mes compétences, tant sur le plan théorique que pratique, dans des domaines variés tels que l'infrastructure DevOps, la conteneurisation, l'orchestration de pipelines CI/CD et l'administration système en environnement sécurisé. La possibilité de contribuer activement à un projet mêlant développement logiciel, optimisation d'infrastructure, automatisation et calcul haute performance correspondait parfaitement à mes attentes. Ce travail m'a permis de me positionner en tant qu'ingénieur pleinement impliqué dans la conception et le déploiement d'une infrastructure critique, et non comme un simple exécutant. J'ai particulièrement apprécié l'équilibre entre autonomie et encadrement, qui m'a permis d'explorer différentes solutions techniques (Docker vs Podman, Docker Compose vs Usernetes) et de prendre des décisions d'architecture en lien avec les objectifs du projet et les contraintes de sécurité du CEA. En outre, j'ai eu la chance d'être très bien accueilli par les équipes du LCAN et de bénéficier d'une intégration fluide dans les parcours proposés par le CEA aux nouveaux arrivants. J'ai également pu échanger avec plusieurs équipes techniques du site de Saclay, notamment l'équipe RSI de l'ISAS et l'équipe DeepLab, ce qui m'a ouvert sur d'autres problématiques d'infrastructure et de sécurité en environnement de recherche.

\bigskip

Je remercie particulièrement mes encadrants M.~Rémi Bougeois et M.~Adrien Bruneton pour leur disponibilité, leur soutien technique tout au long du stage et pour leurs orientations sur les choix d'architecture. Je tiens également à remercier l'ensemble de l'équipe du LCAN pour leur accueil et leur patience face à mes nombreuses questions sur l'écosystème TRUST, ainsi que l'équipe RSI de l'ISAS pour leur support technique sur les aspects réseau et sécurité. Enfin, je remercie l'équipe DeepLab pour la mise à disposition de l'infrastructure GitLab et leur réactivité sur les questions d'administration.
